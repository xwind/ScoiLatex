\begin{problem}{围棋}
近日,谷歌研发的围棋AI—AlphaGo以4:1的比分战胜了曾经的世界冠军李世石,这是人工智能领域的又一里程碑。与传统的搜索式AI不同,AlphaGo使用了最近十分流行的卷积神经网络模型。

在卷积神经网络模型中,棋盘上每一块特定大小的区域都被当做一个窗口。例如棋盘的大小为$5\times 6$,窗口大小为$2\times 4$,那么棋盘中共有12个窗口。此外,模型中预先设定了一些模板,模板的大小与窗口的大小是一样的。下图展现了一个$5\times 6$的棋盘和两个$2\times 4$的模板。对于一个模板,只要棋盘中有某个窗口与其完全匹配,我们称这个模板是被激活的,否则称这个模板没有被激活。例如图中第一个模板就是被激活的,而第二个模板就是没有被激活的。

\begin{figure}[h!]
\centering
\epsfig{figure=prrr.eps,width=10cm}
\end{figure}

我们要研究的问题是:对于给定的模板,有多少个棋盘可以激活它。  

为了简化问题,我们抛开所有围棋的基本规则,只考虑一个$n\times m$的棋盘,每个位置只能是黑子、白子或无子三种情况,换句话说,这样的棋盘共有$3^{n\times m}$种。此外,我们会给出q个$2\times c$的模板。我们希望知道,对于每个模板,有多少种棋盘可以激活它。

强调:模板一定是两行的。

  
\InputFile
输入数据的第一行包含四个正整数$n$,$m$,$c$和$q$,分别表示棋盘的行数、列数、模板的列数和模板的数量。

随后$2\times q$行,每连续两行描述一个模板。其中,每行包含c个字符,字符一定是‘W’,‘B’或‘X’中的一个,表示白子、黑子或无子三种情况的一种。

\OutputFile
应包含q行,每行一个整数,表示符合要求的棋盘数量。由于答案可能很大,你只需要输出答案对$1,000,000,007$取模后的结果即可。

\Example
\begin{example}
\exmp{
	3 1 1 2
B
W
B
B

}{
	6
5 

}
\exmp{
3 3 2 3
XB
BW
BX
XB
BB
BB
}{
963
954
857
}
\end{example}

\Hint
能激活BW的棋盘有:BWB, BWW, BWX, BBW, WBW, XBW共6个

能激活BB的棋盘有:BBB, BBW, BBX, WBB, XBB共5个

\Note
\begin{center}
\begin{tabular}{c|c}
\hline
\textsf{测试点编号}&\textsf{约定}\\
\hline
1	&n=3,m=4,c=2\\
\hline
2	&n=4,m=4,c=3\\
\hline
3	&n=2,m=9,c=6\\
\hline
4	&n=2,m=12,c=3\\
\hline
5	&n=2,m=12,c=5\\
\hline
6	&n=10,m=8,c=3\\
\hline
7	&n=10,m=10,c=5\\
\hline
8	&n=100,m=10,c=5\\
\hline
9	&n=100 m=12,c=5\\
\hline
10	&n=100,m=12,c=6\\
\hline
\end{tabular}
\end{center}
对于所有测试点,$q\le 5$。
\end{problem}

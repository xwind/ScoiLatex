\begin{problem}{幸运数字}
A国共有n座城市,这些城市由$n-1$条道路相连,使得任意两座城市可以互达,且路径唯一。每座城市都有一个幸运数字,以纪念碑的形式矗立在这座城市的正中心,作为城市的象征。

一些旅行者希望游览A国。旅行者计划乘飞机降落在x号城市,沿着x号城市到y号城市之间那条唯一的路径游览,最终从y城市起飞离开A国。在经过每一座城市时,游览者就会有机会与这座城市的幸运数字拍照,从而将这份幸运保存到自己身上。

然而,幸运是不能简单叠加的,这一点游览者也十分清楚。他们迷信着幸运数字是以异或的方式保留在自己身上的。例如,游览者拍了3张照片,幸运值分别是5,7,11,那么最终保留在自己身上的幸运值就是9(5 xor 7 xor 11)。有些聪明的游览者发现,只要选择性地进行拍照,便能获得更大的幸运值。例如在上述三个幸运值中,只选择5和11,可以保留的幸运值为14。

现在,一些游览者找到了聪明的你,希望你帮他们计算出在他们的行程安排中可以保留的最大幸运值是多少。

\InputFile
第一行包含2个正整数$n$、$q$,分别表示城市的数量和旅行者数量。
第二行包含n个非负整数,其中第i个整数$G_i$表示i号城市的幸运值。
随后$n-1$行,每行包含两个正整数x、y,表示x号城市和y号城市之间有一条道路相连。
随后q行,每行包含两个正整数x、y,表示这名旅行者的旅行计划是从x号城市到y号城市。

\OutputFile
输出需要包含q行,每行包含1个非负整数,表示这名旅行者可以保留的最大幸运值。

\Example
\begin{example}
\exmp{
4 2
11 5 7 9
1 2
1 3
1 4
2 3
1 4
}{
14
11
}
\end{example}

\Hint
内存使用不要超过题目限制!

内存使用不要超过题目限制!

内存使用不要超过题目限制!

\Note
\begin{center}
\begin{tabular}{c|c|c}
\hline
\textsf{编号}&\textsf{约定}&\textsf{性质}\\
\hline
1	&n=100,q=100,$G_i<2^{20}$&\multirow{2}{*}{无}\\
\cline{1-2}
2	&n=1000,q=1000,$G_i<2^{30}$\\
\hline
3	&n=10000,q=50000,$G_i<2^{30}$&\multirow{3}{*}{存在两个城市距离为$n-1$}\\
\cline{1-2}
4	&n=20000,q=50000,$G_i<2^{60}$\\
\cline{1-2}
5	&n=20000,q=100000,$G_i<2^{60}$\\
\hline
6	&n=10000,q=50000,$G_i<2^{30}$&\multirow{5}{*}{无}\\
\cline{1-2}
7	&n=20000,q=50000,$G_i<2^{60}$\\
\cline{1-2}
8	&n=20000,q=80000,$G_i<2^{60}$\\
\cline{1-2}
9	&n=20000,q=120000,$G_i<2^{60}$\\
\cline{1-2}
10	&n=20000,q=200000,$G_i<2^{60}$\\
\hline
\end{tabular}
\end{center}
\end{problem}

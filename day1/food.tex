\begin{problem}{美味}
一家餐厅有n道菜,编号1$\dots$n,大家对第i道菜的评价值为$a_i (1 \leq i \leq n)$。

有m位顾客,第i位顾客的期望值为$b_i$, 而他的偏好值为$x_i$。

因此,第i位顾客认为第j道菜的美味度为$b_i\ XOR\ (a_j + x_i)$, XOR表示异或运算。

第i位顾客希望从这些菜中挑出他认为最美味的菜,即美味值最大的菜,但由于价格等因素,他只能从第$l_i$道到第$r_i$道中选择。

请你帮助他们找出最美味的菜。

\InputFile
第1行, 两个整数, $n,\ m$, 表示菜品数和顾客数。

第2行, n个整数, $a_1,\ a_2,\ \cdots,\ a_n$,表示每道菜的评价值。

第3至m+2行, 每行4个整数, $b,\ x,\ l,\ r$, 表示该位顾客的期望值,偏好值, 和可以选择菜品区间。

\OutputFile
输出m行,每行1个整数,$y_{max}$, 表示该位顾客选择的最美味的菜的美味值。

\Example
\begin{example}
\exmp{
4 4
	1 2 3 4
	1 4 1 4
	2 3 2 3
	3 2 3 3
	4 1 2 4
}{
	9
	7
	6
	7
}%
\end{example}

\Note
其中, $1 \leq n \leq 2\times 10^5,\ 0 \leq a_i,\ b_i,\ x_i\ <\ 10 ^ 5,\ 1\leq l_i \leq r_i \leq n (1 \leq i \leq m)$
	
对于30\%的数据, $1 \leq  m \leq 10^3$
	
对于100\%的数据, $1 \leq m \leq 10^5$
\end{problem}

